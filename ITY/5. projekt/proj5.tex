\documentclass[unicode, breaklinks]{beamer}
\usepackage[utf8]{inputenc}
\usepackage[czech]{babel}
\usepackage{csquotes}
\usepackage{listings}
\usetheme{Antibes}

\title{Typografie a publikování\,--\,5. projekt \\
       Vytváření prezentací v LaTeXu}
\author{Lukáš Baštýř (xbasty00)}
\institute{Vysoké učení technické v Brně\\
           Fakulta informačních technologií}
\date{\today}

\begin{document}

    \frame{\titlepage}
    
    \title{Datové struktury - zásobník}
    
    \frame{\titlepage}

    \frame{\tableofcontents}

    \section{Úvod}
    \begin{frame}
        \frametitle{Úvod}
        Existuje velké množství datových typů. Zásobník je ale jeden z těch jednoduších a zároveň asi nejvíce užíváných.
    \end{frame}
    
    \section{Základní popis}
    \begin{frame}
        \frametitle{Zásobník}
        \begin{itemize}
            \item Jedná se o lineární datový typ
            \item Data která byla uložena jako poslední se čtou jako první\\ \textbf{LIFO} (z angl. \emph{last in, first out})
            \item Použití:
            \begin{itemize}
                \item Undo/redo operace v různých editorech
                \item Otočení pořádí prvků v datové struktuře
                \item Výpočet matematických operací
            \end{itemize}
        \end{itemize}
    \end{frame}

    \section{Základní operace}
        \begin{frame}
            \frametitle{Základní operace na zásobníku}
            \begin{itemize}
                \item Init\,-\,inicializace zásobníku
                \item Push\,-\,vložení položky na vrchol zsobníku
                \item Pop\,-\,vyjmutí poslední přídané položky ze zásobníku 
                \item Top\,-\,získání vrchní prvku na zásobníku (bez jeho odstranení ze     zásobníku)
        \item Is\_empty\,-\,dotaz, zda je zásobník prázdný
    \end{itemize}
    \end{frame}

    \section{Implementace v pseudokódu}
    \begin{frame}[fragile]
        \frametitle{Příklad implementace}
        \begin{lstlisting}[basicstyle=\small]
stack test_s;    //create new stack
test_s.init();   //initialize empty stack
test_s.push(5);  //push 5 onto the stack
test_s.push(10); //push 10 onto the stack
int last_var = test_s.pop(); //take last value from
                             //stack (last_var = 10)
bool empty = test_s.empty(); //empty is now false as
                             //stack is not empty yet
int top_var = test_s.top();  //top_var = 5, because
                             //10 is top variable
top_var = test_s.pop(); //top_var is again 5, but
                        //stack should be now empty
empty = test_s.empty(); //empty is now true, as
                        //stack is actually empty
        \end{lstlisting}
    \end{frame}

\end{document}
