\section{Příklad 5}
% Jako parametr zadejte skupinu (A-H)
\patyZadani{E}

Sestavíme rovnice pro smyčková napětí v obvodu a vyjádříme proud:
    \begin{eqnarray*}
        U &= & U_{C} + U_{R}\\
        U_{R} &= & R * I\\
        U &= & U_{C} + R * I\\
        \\
        I &= & \frac{U - U_{C}}{R}\\
	\end{eqnarray*}

Hodnoty dosadíme do rovnice (axiomu) pro tento obvod a upravíme:
    \begin{eqnarray*}
        U_{C}' &= & \frac{1}{C}I\\
        U_{C}' &= & \frac{1}{C} * \frac{U - U_{C}}{R}\\
        U_{C}' &= & \frac{U - U_{C}}{RC}\\
        \\
        U_{C}' &= & \frac{40 - U_{C}}{40 * 30} = \frac{1}{30} - U_{C}\frac{1}{1200}\\
        U_{C}' + U_{C}\frac{1}{1200} &= & \frac{1}{30}\\
	\end{eqnarray*}

Obecný tvar rovnice a výpočet $\lambda$:
    \begin{eqnarray*}
        U_{C}(t) &= & K(t)e^{\lambda t}\\
        0 &= & \lambda + \frac{1}{1200}\\
        \lambda &= & -\frac{1}{1200}\\
	\end{eqnarray*}

Dosadíme do obecného a zderivovaného tvaru:
    \begin{eqnarray*}
        U_{C}(t) &= & K(t)e^{-\frac{1}{1200}t}\\
        U_{C}'(t) &= & K'(t)e^{-\frac{1}{1200}t} -\frac{1}{1200}K(t)e^{-\frac{1}{1200}t}\\
	\end{eqnarray*}

Dále dosadíme do rovnice obvodu:
    \begin{eqnarray*}
        U_{C}' + U_{C}\frac{1}{1200} &= & \frac{1}{30}\\
        K'(t)e^{-\frac{1}{1200}t} -\frac{1}{1200}K(t)e^{-\frac{1}{1200}t} + \frac{1}{1200}K(t)e^{-\frac{1}{1200}t} &= & \frac{1}{30}\\
        K'(t)e^{-\frac{1}{1200}t} &= & \frac{1}{30}\\
	\end{eqnarray*}

Zjistíme $K(t)$ (integrováním):
    \begin{eqnarray*}
        K'(t)e^{-\frac{1}{1200}t} &= & \frac{1}{30}\\
        K'(t) &= & \frac{e^\frac{1}{1200}t}{30}\\
        K(t) = \int K'(t) dt &= & \frac{e^\frac{1}{1200}t}{30}\\
        K(t) &= & 40e^\frac{1}{1200}t + X\\
	\end{eqnarray*}

Dosadíme do obecného tvaru:
    \begin{eqnarray*}
        U_{C}(t) &= & K(t)e^\lambda t = (40e^\frac{1}{1200} + X)e^{-\frac{1}{1200}t} = 40 + Xe^{-\frac{1}{1200}t}\\
	\end{eqnarray*}

Zjistíme C pomocí počáteční podmínky:
    \begin{eqnarray*}
        U_{C}(0) &= & 40 + Xe^{-\frac{1}{1200}t}\\
        U_{C}(0) &= & 40 + Xe^{-\frac{1}{1200} * 0}\\
        11 &= & 40 + X\\
        X &= & - 29\\
	\end{eqnarray*}

Naposledy dosadíme:
    \begin{eqnarray*}
        U_{C}(t) &= &  40 + Xe^{-\frac{1}{1200}t}\\
        U_{C}(t) &= &  40 + 29e^{-\frac{1}{1200}t}\\
	\end{eqnarray*}

Hledaná hodnota $U_{C}(t)$ je:
    \begin{eqnarray*}
        U_{C}(t) &= & 40 - 29e^{-\frac{1}{1200}t}\\
	\end{eqnarray*}
	
Kontrola:
    \begin{eqnarray*}
        U_{C}' + U_{C}\frac{1}{1200} &= & \frac{1}{30}\\
        U_{C}(t) &= & 40 - 29e^{-\frac{1}{1200}t}\\
        U_{C}'(t) &= & \frac{29}{1200}e^{-\frac{1}{1200}t}\\
        \frac{29}{1200}e^{-\frac{1}{1200}t} + \frac{1}{1200}(40 - 29e^{-\frac{1}{1200}t}) &= & \frac{1}{30}\\
        \frac{29}{1200}e^{-\frac{1}{1200}t} - \frac{29}{1200}e^{-\frac{1}{1200}t} + \frac{1}{30} &= & \frac{1}{30}\\
        \frac{29}{1200}e^{-\frac{1}{1200}t} + \frac{1}{30} &= & \frac{1}{30} + \frac{29}{1200}e^{-\frac{1}{1200}t}\\
        0 &= & 0\\
	\end{eqnarray*}