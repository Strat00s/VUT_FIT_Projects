\section{Příklad 4}
% Jako parametr zadejte skupinu (A-H)
\ctvrtyZadani{C}

Vypočítáme kapacitanci:
    \begin{eqnarray*}
        X_{C} &= & \frac{-i}{2\pi fC} = - \frac{1}{2 \pi fC}i\\
        X_{C1} &= & - \frac{1}{2 \pi * 75 * 230 * 10^{-6}}i = -9.2264i \Omega\\
        X_{C2} &= & - \frac{1}{2 \pi * 75 * 85 * 10^{-6}}i = -24.9655i \Omega\\
	\end{eqnarray*}

Vypočítáme induktanci:
    \begin{eqnarray*}
        X_{L} &= & 2 \pi fLi\\
        X_{L1} &= & 2 \pi * 75 * 220 * 10^{-3} = 103.6726i \Omega\\
        X_{L2} &= & 2 \pi * 75 * 70 * 10^{-3} = 32.9867i \Omega\\
	\end{eqnarray*}

Sestavíme rovnice pro smyčková napětí v obvodu:
    \begin{eqnarray*}
        A &: & 0 = U_{1} + I_{A}X_{L1} + I_{A}X_{C2} - I_{C}X_{C2}\\
        B &: & 0 = U_{2} + I_{B}X_{C1} + I_{B}R_{1}\\
        C &: & 0 = -U_{2} + I_{C}X_{C2} + I_{C}R_{2} + I_{C}X_{L2} - I_{A}X_{C2}\\
	\end{eqnarray*}

Upravíme rovnice:
    \begin{eqnarray*}
        - U_{1} &= & I_{A}(X_{L1} + X_{C2}) + 0I_{B} - I_{C}X_{C2}\\
        - U_{2} &= & 0I_{A} + I_{B}(X_{C1} + R_{1}) + 0I_{C}\\
        U_{2} &= & -I_{A}X_{C2} + 0I_{B} + I_{C}(X_{L2} + X_{C2} + R_{2})\\
	\end{eqnarray*}
	
Dosadíme do rovnic:
    \begin{eqnarray*}
        -35 &= & 78.7071iI_{A} + 0I_{B} + 24.9655iI_{C}\\
        -45 &= & 0I_{A} + (10 - 9.2264i)I_{B} + 0I_{C}\\
        45 &= & 24.9655iI_{A} + 0I_{B} + (13 + 8.0212i)I_{C}\\
	\end{eqnarray*}

Zapíšeme jako matici a vypočítáme determinant:
    \begin{eqnarray*}
        A &= &
			\begin{pmatrix}			
				78.7071i & 0 & 24.9655i & \vline & -35\\
				0 & 10 - 9.2264i & 0 & \vline & -45\\
				24.9655i & 0 & 13 + 8.0212i & \vline & 45\\
			\end{pmatrix} \\
		\\
		Det A &= &
			\begin{vmatrix}
				78.7071i & 0 & 24.9655i\\
				0 & 10 - 9.2264i & 0\\
				24.9655i & 0 & 13 + 8.0212i \\
			\end{vmatrix} = 9359.8894 + 10306.1881i\\
		\\
	\end{eqnarray*}

Vypočítáme z matice $I_{A}$ a $I_{C}$:
    \begin{eqnarray*}
        I_{A} &= &
			\begin{vmatrix}
				-35 & 0 & 24.9655i\\
				-45 & 10 - 9.2264i & 0\\
				45 & 0 & 13 + 8.0212i\\
			\end{vmatrix} / Det A = - 1.3688 + 0.4555i A\\
		\\
		I_{C} &= &
			\begin{vmatrix}
				78.7071i & 0 & -35\\
				0 & 10 - 9.2264i & -45\\
				24.9655i & 0 & 45\\
			\end{vmatrix} / Det A = 4.3153 - 0.0339i A\\
		\\
	\end{eqnarray*}

Vypočítáme hledané hodnoty:
    \begin{eqnarray*}
        I_{C2} &= & I_{C} - I_{A} = (4.3153 - 0.0339i) - (- 1.3688 + 0.4555i) = 5.6841 - 0.4894i\\
        U_{C2} &= & I_{C2} * X_{C2} = (5.6841 - 0.4894i) * - 24.9655i = - 12.2181 - 141.9064i\\
        |U_{C2}| &= & \sqrt{RealU_{C2}^2 + ImgU_{C2}^2} = \sqrt{(-12.2181)^2 + (-141.9064)^2} = 142.4314 V\\
        \end{eqnarray*}
               3. kvadrant
    \begin{eqnarray*}
        \varphi_{C2} &= & \pi + tan^-1(\frac{ImgU_{C2}}{RealU_{C2}})\\
        \varphi_{C2} &= & \pi +  tan^-1(\frac{-141.9064}{-12.2181}) = 4.6265 rad\\
        \varphi_{C2} &= & \frac{180}{\pi} * 4.6265 rad = \ang{265.07899}\\
	\end{eqnarray*}

Hledané hodnoty $|U_{C2}|$ a $\phi_{C2}$ jsou:
    \begin{eqnarray*}
        |U_{C2}| = 142.4314 V\\
        \varphi_{C2} = \ang{265.07899}\\
	\end{eqnarray*}