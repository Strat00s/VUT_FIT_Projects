\section{Příklad 3}
% Jako parametr zadejte skupinu (A-H)
\tretiZadani{A}

I. Kirchhův zákon použijeme na uzly v obvodu:
    \begin{eqnarray*}
        A&: & 0 = I_{R1} + I_{R2} - I_{1}\\
        B&: & 0 = I_{R4} -I_{R2} - I_{R5}\\
        C&: & 0 = I_{R3} + I_{R5} -I_{R4} - I_{2}\\
	\end{eqnarray*}

Vyjádříme proudy pomocí úzlových napětí:\\
    
               Substituce $G_{x} = \frac{1}{R_{x}}$
    \begin{eqnarray*}
        I_{R1} &= & \frac{U_{A}}{R_{1}} = G_{1} * U_{A}\\
        I_{R2} &= & \frac{U_{A} - U_{B}}{R_{2}} = G_{2}(U_{A} - U_{B})\\
        I_{R3} &= & \frac{U_{C}}{R_{3}} = G_{3} * U_{C}\\
        I_{R4} &= & \frac{U_{A} - U_{B}}{R_{4}} = G_{4}(U_{A} - U_{B})\\
        I_{R5} &= & \frac{U + U_{C} - U_{B}}{R_{5}} = G_{5}(U + U_{C} - U_{B})\\
	\end{eqnarray*}

Dosadíme proudy do rovnic:
    \begin{eqnarray*}
        0 &= & G_{1}U_{A} + G_{2}(U_{A} - U_{B}) - I_{1}\\
        0 &= & G_{4}(U_{B} - U_{C}) - G_{2}(U_{A} - U_{B}) - G_{5}(U + U_{C} - U_{B})\\
        0 &= & G_{3}U_{C} + G_{5}(U + U_{C} - U_{B}) - G_{4}(U_{B} - U_{C}) - I_{2}\\
	\end{eqnarray*}
	
Upravíme rovnice:
    \begin{eqnarray*}
        I_{3} &= & G_{5}U\\
        \\
        I_{1} &= & U_{A}(G_{1} + G_{2}) - U_{B}(G_{2})\\
        I_{3} &= & U_{B}(G_{2} + G_{4} + G_{5})-U_{A}(G_{2}) - U_{C}(G_{4} + G_{5})\\
        I_{2} - I_{3}&= & U_{C}(G_{3} + G_{4} + G_{5}) - U_{B}(G_{4} + G_{5})\\
	\end{eqnarray*}
	
Dosadíme hodnoty:
    \begin{eqnarray*}
        0.9 &= & 0.0393U_{A} - 0.0204U_{B} + 0 U_{C}\\
        3.75 &= & -0.0204U_{A} + 0.0773U_{B} - 0.0569U_{C}\\
        -3.05 &= & 0U_{A} - 0.0569U_{B} + 0.0723U_{C}\\
	\end{eqnarray*}

Zapíšeme jako matici a vypočítáme determinant:
    \begin{eqnarray*}
        A &= &
			\begin{pmatrix}			
				0.0393 & - 0.0204 & 0 & \vline & 0.9 \\
				- 0.0204 & 0.0773 & - 0.0569 & \vline & 3.75 \\
				0 & - 0.0569 & 0.0723 & \vline & -3.05
			\end{pmatrix} \\
		\\
		Det A &= &
			\begin{vmatrix}
				0.0393 & - 0.0204 & 0\\
				- 0.0204 & 0.0773 & - 0.0569\\
				0 & - 0.0569 & 0.0723 \\
			\end{vmatrix} = 6.2313 * 10^{-5}\\
		\\
	\end{eqnarray*}

Vypočítáme z matice $U_{B}$ a $U_{C}$:
    \begin{eqnarray*}
        U_{B} &= &
            \begin{vmatrix}
				0.0393 & 0.9 & 0\\
				- 0.0204 & 3.75 & - 0.0569\\
				0 & - 3.05 & 0.0723 \\
			\end{vmatrix} / Det A = 82.845 V\\
		\\
        U_{C} &= &
            \begin{vmatrix}
				0.0393 & - 0.0204 & 0.9\\
				- 0.0204 & 0.0773 & 3.75\\
				0 & - 0.0569 & -3.05 \\
			\end{vmatrix} / Det A = 23.0136 V\\
		\\
	\end{eqnarray*}

Dopočítáme hledané hodnoty:
    \begin{eqnarray*}
        U_{R4} &= & U_{B} - U_{C} = 82.845 - 23.0136 = 59.8314 V\\
        I_{R4} &= & \frac{U_{R4}}{R_{4}} = \frac{58.8314}{39} = 1.5341 A\\
	\end{eqnarray*}
	
Hledané hodnoty $U_{R4}$ a $I_{R4}$ jsou:
    \begin{eqnarray*}
        U_{R4} &= & 59.8314 V\\
        I_{R4} &= & 1.5341 A\\
	\end{eqnarray*}